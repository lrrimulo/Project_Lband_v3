\documentclass{article}
\usepackage{graphicx}
\usepackage{amsmath}

\begin{document}

%\title{Introduction to \LaTeX{}}
%\author{Author's Name}
%
%\maketitle

Hola Cesar, 

Yo tuvo una idea sobre como estimar el error en una cantidad que es integrada. Primero, recordamos que una integral puede ser aproximada por trapezios, como se sigue:
%
\begin{align}
B 
&=
\int_{\lambda_1}^{\lambda_2}F\mathrm{d}\lambda\notag\\
&\approx 
\sum_{i=1}^{N-1}\frac{1}{2}\left(F_{i+1}+F_i\right)\Delta\lambda\notag\\
&\approx
\sum_{i=1}^{N}F_i\Delta\lambda\,.
\label{eq1}
\end{align}
%


Ahora, para una funci\'on $f$ que depende de $x,y,...$, el error $\sigma_f$ se propaga de acuerdo con $\sigma_f^2 = \left(\frac{\partial f}{\partial x}\right)^2\sigma_x^2+\left(\frac{\partial f}{\partial y}\right)^2\sigma_y^2+...$. As\'\i, el error de $B$, dado por la ecuaci\'on \ref{eq1} es
%
\begin{align}
\sigma_B^2
&\approx 
\sum_{i=1}^{N}(\Delta\lambda)^2\sigma_{F_i}^2\,.
\end{align}
%

Simplificando m\'as un poco, consideramos que los errores en cada punto son iguales: $\sigma_{F_i}\approx \sigma_{F_i} \forall i$. As\'\i, la ecuaci\'on acima se torna:
%
\begin{align}
\sigma_B^2
&\approx 
N(\Delta\lambda)^2\sigma_{F}^2 = \frac{1}{N}(N\Delta\lambda)^2\sigma_{F}^2
\end{align}
%
donde podemos ver que $N\Delta\lambda = (\lambda_2-\lambda_1)$ y 
%
\begin{align}
\sigma_F^2=\left\langle(\Delta F)^2\right\rangle=\frac{1}{N}\sum_{i=1}^N(\Delta F)_i^2\,,
\end{align}
%
donde $(\Delta F)_i$ son las distancias entre cada punto del espectro y la funci\'on polinomial que tu ajustaste. 


As\'\i, la expresi\'on final para el error es:
%
\begin{align}
\sigma_B \approx \frac{1}{N}(\lambda_2-\lambda_1)\left(\sum_{i=1}^N(\Delta F)_i^2\right)^\frac{1}{2}
\end{align}
%

Que te parece? :)



\end{document}

